\documentclass[11pt,a4paper,draft]{report}
\usepackage[utf8]{inputenc}
\usepackage{amsmath}
\usepackage{amsfonts}
\usepackage{amssymb}
\usepackage{graphicx}
\usepackage[final]{listings}
\usepackage{url}
\author{S. Shiraiwa}
\title{Petra-M: Physics Equation Translator for MFEM}
\begin{document}
\lstset{language=Python}
\maketitle
\tableofcontents
\newpage

\chapter{Introduction}
Petra-M (Physics Equation Translator for MFEM) is a tool to build a finite element simulation model using MFEM. 
MFEM is a scalable finite element library built at LLNL. In MFEM,  a user has to fill the linear system matrix and the right hand side by adding  mfem::BilinearFormIntegrator and mfem::LinearFormIntegrator to mfem::BilinearForm and mfem::LinearForm.
While a variety of integrators are already defined in MFEM, translating a physics equation to a weakform and choosing a proper integrator for each case is an error-prone process. 
Another practical issue is to assign domain and boundary conditions for a particular element. A real world 3D geometry could contain several handreds or even more surfaces and domains. Without an interactive interface, it is difficult to perform this step in a reliable manner.

Using Petra-M and $\pi$Scope, a user can build physics simulation model quickly and reliably. Petra-M currently support only frequency domain Maxwell problems and simple thermal diffusion model. However, its low level engine is design to be flexible to expand in future.

Goal of this report is to describe how Petra-M construct a linear system. An emphasis is on showing the weakform equation systems used for each physics module. Note that these equations are mostly well established and found in various literature, and therefore, giving detailed derivation is not the point of this report. 


\chapter{Preparing simulation model}
\section{Linear System Structure}
Need to discusss,,,,Mesh space, FESpace variable, Auxiliary variable

\section{Namespace and Functional Coefficients}
Using it for material property and selection index.
\subsection{Namespace}

\subsection{Function}
\begin{minipage}[c]{0.95\textwidth}
\begin{lstlisting}[caption={A user defined generic preconditioner},captionpos=b, frame=single, label={function1}]

from petram.helper.variables import variable

@variable.float()

@variable.array(complex = False, shape=(3,3), \
               dependency=['vhts1', 's'])
def func(x, y, z, vhts1=None, s=None):

     # compute your material
     return np.array([1,0,0], [0,1,0], [0,0,1]])

\end{lstlisting}
 \end{minipage}

\section{Auxiliary Variable and Operator}
\subsection{Operator}
\subsubsection{identity}            
\subsubsection{integral}
\subsubsection{projection}
\subsubsection{deltam}
\subsubsection{deltam}
deltam operator is an extention of delta operator. Deltam operator has a following argument interface.


Using it for material property and selection index.

\chapter{PDE modules}
\section{WeakForm(WF)}
\section{CoefficientForm(Coeff)}
\subsection{2D static (Coeff2D)}
This module solves the following equation in 2D (x, y) domain.
  \begin{align}
   \nabla (\mathbf{c} \nabla u + \mathbf{p}u + \mathbf{r}) + \mathbf{q} \cdot \nabla u + au - f = 0 
  \\  
  u = u_{0} \,\,\,on\,\,\,\partial \Omega_{1}
  \\
 \mathbf{n}\cdot (\mathbf{c} \nabla u + \mathbf{p}u + \mathbf{r}) = hu + g,\,\,on\,\,\,\partial \Omega_{2}
  \end{align}
  
 

 
  \subsection{2D time-dependent (Coeff2Dt)}
  \begin{align}
  m\frac{\partial^2 u}{\partial t^2} +   d\frac{\partial u}{\partial t} + \nabla (\mathbf{c} \nabla u + \mathbf{p}u + \mathbf{r}) + \mathbf{q} \cdot \nabla u + au - f = 0, 
  \\  
  u = u_{0} \,\,\,on\,\,\,\partial \Omega_{1},
  \\
 \mathbf{n}\cdot (\mathbf{c} \nabla u + \mathbf{p}u + \mathbf{r}) = hu + g,\,\,on\,\,\,\partial \Omega_{2},
  \end{align}
where m is the mass coefficient, d is the damping coefficient, c is the diffusion coefficient, p is the conservative flux convection coefficient. q is the convection coefficient,  a is the absorption coefficient, r is the conservative flux source term, and f is the source term.

 \subsection{Weakform}
  Multipling a test funcion $v$ from the left and integrating it over the computation domain,  we can transform the terms in the previous 2nd order PDE as follows.
  \begin{equation}
 \int v \nabla (\mathbf{c} \cdot \nabla u) = (\nabla v,\mathbf{c} \cdot \nabla u) - \langle v, \mathbf{c} \cdot \frac{\partial u}{\partial \mathbf{n}} \rangle
  \end{equation}
  
  a u + gamma) 
              + beta (grad u) + a u - f = 0

  On domain boundary
     n ( c grad u + alpha u - gamma) + q u = g - h$^t$ mu
       or 
     u = u0  

    m, d, a, f, g and h: scalar
    alpha, beta and gamma : vector
    c  : matrix (dim (space) $^2$)

    If all coefficients are independent from u, ux,
    the system is linear.

    BC
     Zero Flux : 
        n ( c grad u + alpha u - gamma) = 0
     Flux: 
        n ( c grad u + alpha u - gamma) = - g + q u
     Dirichlet Boundary Condition
        u = u0

  Weakform integrators:
    domain integral
       c     -->  c (grad u, grad v)     bi
       alpha -->  alpha * (u, grad v)    bi
       gamma -->  gamma * grad v         li
       beta  -->  beta * (grad u, v)     bi
       f     -->  (f, v)                 bi
    surface integral
       c     -->   n dot c (grad u, v)   bi
       alpha -->   n dot alpha  (u v)    bi
       gamma -->   n dot gamma   v       li
 
    surface integral can be replaced by (g - qu, v)
        
  Domain:   
     Coeff2D          : tensor dielectric

  Boundary:
     Coeff2D-Zero     : zero essential (default)
     Coeff2D-Esse     : general essential

'''
\section{Radio Frequency (EM)}

We employ $\exp{(-\omega t)}$ for time dependence. Then, from the Maxwell's equations, it follows,
\begin{equation}
\nabla \times \frac{1}{\mu_0} \nabla \times \mathbf{E} - (\omega^2 \epsilon + i \omega \sigma) \mathbf{E} = i \omega \mathbf{J}_{\rm ext}
\end{equation}

\subsection{3D frequency domain(EM3D)}
\subsubsection{Weakform of the Maxwell's equation}
This module uses the Cartesian coordinate system (x, y, z), and solves the following weakform of Maxwell equations. 
 \begin{align}
(\nabla \times \mathbf{F},  \frac{1}{\mu}\nabla  \times  \mathbf{E})
 - (\mathbf{F},  (\omega^2 \epsilon+ i \omega \sigma)  \mathbf{E}) 
 +  \langle \mathbf{F},  \mathbf{Q} \rangle 
 \notag \\
 - \gamma \langle  \mathbf{F}, \mathbf{n} \times \mathbf{n} \times  \mathbf{E}\rangle = i \omega (\mathbf{F}, \mathbf{J}_{\rm ext} ) ,
\label{em3d_weak} \\
\label{em3d_weakBC} 
 \mathbf{n} \times (\frac{1}{\mu} \nabla \times \mathbf{E}) + \gamma \mathbf{n} \times \mathbf{n} \times  \mathbf{E} = Q\,\,\,on\,\,\,\partial \Omega_{2}
 \end{align}
  where $(A , B)$ is the domain integral ($\int_{\Omega} AB dxdydz$) and $\langle A, B \rangle $ is the boundary integral ($\int_{\partial \Omega_{2}} ABdxdydz$). 
 
 \subsubsection{Anisotropic media}
 
 \subsubsection{External current source}
 
 \subsubsection{PMC (perfect magnetic conductor)}
 
 \subsubsection{PEC (perfect electric conductor/electric field BC}
 
  \subsubsection{Surface Current}
  
 \subsubsection{Port}
 Port can be used  on the boundary $\partial\Omega$, where the mode profile ($\mathbf{E}_{\rm mode}$ and $\mathbf{H}_{\rm mode}$) and the amplitude incoming mode ($A_{\rm inc}$) is known. The outgoing wave amplitude ($A_{\rm out}$) is not known and is solved as a part of equation. We use $\gamma=0$ and $\frac{1}{\mu}\nabla\times\mathbf{E}=i \omega \mathbf{H}$ in Eq.~\ref{em3d_weak} and obtain a linear system in the block matrix from,
\begin{equation}
\label{port_BC_blockmatrix}
 \begin{bmatrix}
 \mathbf{S}   & -\mathbf{N} \\
\mathbf{M}^t & 1\\
\end{bmatrix}
 \begin{bmatrix}
 \mathbf{x} \\
A_{\rm out}\\
\end{bmatrix}
=
 \begin{bmatrix}
A_{\rm inc} \mathbf{N}     \\
0 \\
\end{bmatrix},
 \end{equation}
where $\mathbf{N}$ is the linearform to impose  $\mathbf{H}_{\rm mode}$ using Eq.~\ref{em3d_weakBC} and is written as
\begin{equation}
\mathbf{N} = \langle \mathbf{F}, i \omega \mathbf{n} \times \mathbf{H}\rangle 
 \end{equation}
 
$\mathbf{M}^t$ is the weight to extract the mode amplitude from the solution vector $\mathbf{x}$. In the continuous form, it corresponds to the following 
\begin{equation}
\int \mathbf{E} \cdot \mathbf{E}_{\rm mode} \partial\Omega/\int \mathbf{E}_{\rm mode}  \cdot \mathbf{E}_{\rm mode} \partial\Omega
\end{equation}

In order to define $\mathbf{M}^t$ in the discrete space , we first the vector representation of $\mathbf{E}_{\rm mode}$ in the H(curl) space. This can be done by projecting the coefficient to GridFunction, and we note it as $\mathbf{E}_{\rm gf}$. Then $\mathbf{M}$ is given by
\begin{equation}
\mathbf{M} = \langle \mathbf{F}, \mathbf{E_{\rm mode}} \rangle / \langle \mathbf{F}, \mathbf{E}_{\rm mode} \rangle \cdot \mathbf{E}_{\rm gf},
 \end{equation}
where the dot product is computed between the second linearform and $\mathbf{E}_{\rm gf}$.

Note that the current implementation assumes that there is only one out-going mode. If the geometry allows to sustain multiple modes, we needs to expand Eq.~\ref{port_BC_blockmatrix} to include multiple modes. 

For the rectangular waveguide and coax port, we normalize the mode amplitude in such a way that the time averaged  incoming wave power is equal to 1[W]. For the coax (TEM) case, the following mode profiles are used.

 \begin{align}
E_{\rm mode} = A_{\rm n} \frac{\mathbf{r}}{r} \exp(kz)
\\
i \omega H_{\rm mode} = A_{\rm n} \omega \sqrt{\frac{\epsilon_{\rm 0}\epsilon_{\rm r}}{\mu_{\rm r}\mu_{\rm 0}}} \frac{\mathbf{r}}{r} \exp(kz)
\\
 A_{\rm n}^2 = \frac{1}{\pi \ln{(b/a)}} \sqrt{\frac{\mu_{\rm r}\mu_{\rm 0}}{\epsilon_{\rm 0}\epsilon_{\rm r}}}
\end{align}

For the waveguide (TE) case, the following mode profiles are used.
(...do be documented...)

Note that we define Poyx, Poyy, and Poyz (x, y and z components of the Poynting vector) as $E \times \overline{B}$, which is twice bigger than the time averaged power flux.


 
 \subsubsection{Periodic (Floquet-Bloch) BC}
 
\subsection{Axisymmetric frequency domain(EM2Da)}
\subsubsection{Weakform of the Maxwell's equation}
This module uses the cylindrical coordinate system (r,  $\phi$, z). Physics quantities are supposed to have a periodic dependence to 
$\phi$ direction ($\sim e^{m \phi}$), where $m$ is the out-of-plane mode number. First, we write the curl operator in the following form

 \begin{align}
 \nabla \times E = &(\frac{im}{r} E_{\rm z} - \frac{\partial E_{\phi}}{\partial z})\mathbf{e}_{\rm r} +
( \frac{\partial E_{\rm r}}{\partial z} - \frac{\partial E_{\rm z}}{\partial r})\mathbf{e}_{\rm \phi}+
 (\frac{1}{r}\frac{\partial }{\partial r}(r E_{\phi}) - \frac{im}{r} E_{\rm r} )\mathbf{e}_{\rm z} 
 \notag \\ 
 = &\nabla_{t} \times \mathbf{E}_{\rm t} + \frac{\mathbf{e}_{\rm \phi}}{r} \times (im \mathbf{E}_{\rm t} - \nabla_{\rm t} (rE_{\rm \phi}) ),
 \end{align}
 where $\mathbf{E}_{\rm t} = (E_{\rm r}, E_{\rm z})$ and  $\nabla_{t} = (\frac{\partial }{\partial r}, \frac{\partial }{\partial z}$) are 2D vectors on $r$, $z$ plane.
 
Then, we exapned the weakform for Maxwell equations (Eq.\,\ref{em3d_weak}) using $\mathbf{E}_{\rm t}$, $E_{\phi}$, and $\nabla_{t}$ so that a direct one-by-one connection to mfem::LinearFormIntegrator and mfem::BilinearFormIntegrator becomes clear.
For a scolar $\mu$, $\epsilon$, and $\sigma$, it can be written as 
 \begin{align}
r(\nabla_{\rm t} \times \mathbf{F}_{\rm t},  \frac{1}{\mu}\nabla_{\rm t}  \times & \mathbf{E}_{\rm t}) 
+\frac{1}{r} m^2(\mathbf{F}_{\rm t}, \frac{1}{\mu}\mathbf{E}_{\rm t})
+ \frac{1}{r} (\nabla_{\rm t} (rF_{\rm \phi}), \frac{1}{\mu}\nabla_{\rm t} (rE_{\rm \phi}))
 \notag \\
+& \frac{1}{r} (im \mathbf{F}_{\rm t}, \frac{1}{\mu}\nabla_{\rm t} (rE_{\rm \phi}))
- \frac{1}{r} (\nabla_{\rm t} (rF_{\rm \phi}), \frac{im}{\mu} \mathbf{E}_{\rm t})
 \notag \\
 -& (\omega^2 \epsilon+ i \omega \sigma) ( r( \mathbf{F}_{\rm t}, \mathbf{E}_{\rm t}) + \frac{1}{r}(rF_{\rm \phi}, rE_{\rm \phi})) 
 \notag \\
 +& r \langle \mathbf{F}_{\rm t},  \mathbf{Q}_{\rm t} \rangle + \langle rF_{\rm \phi}, Q_{\rm phi} \rangle -r \gamma \langle \mathbf{F}_{\rm t}, \mathbf{E}_{\rm t} \rangle
 \notag \\
 =& i \omega r ( \mathbf{F}_{\rm t}, \mathbf{J}_{\rm t} ) + i \omega ( rF_{\rm \phi}, J_{\rm \phi} ), \label{em2da_weak}
 \end{align}
 where $(A , B)$ is the domain integral ($\int_{\Omega} AB drdz$) and $\langle A, B \rangle $ is the boundary integral ($\int_{\partial \Omega} ABdrdz$). The module uses the H(curl) element for $\mathbf{E}_{\rm t}$ and the H1 element for $rE_{\phi}$. Note that the integration does not consider $2 \pi r$ and $r$ is included in the coefficient of the linear/bilinear forms. 
 
 \subsubsection{Anisotropic media}
 

 \begin{equation}
 \left[ {\begin{array}{*{20}c}
\mathbf{F}_{\rm t}  & rF_{\rm \phi}\\
\end{array}}\right]
\left[ {\begin{array}{*{20}c}
\mathbf{\epsilon}_{\perp \perp}  & \frac{\mathbf{\epsilon}_{\perp \parallel}}{r}\\
\frac{\mathbf{\epsilon}_{\parallel \perp}}{r} & \frac{\mathbf{\epsilon}_{\parallel \parallel}}{r^2}\\
\end{array}}\right]
 \left[ {\begin{array}{*{20}c}
\mathbf{E}_{\rm t}  \\
rE_{\rm \phi}\\
\end{array}}\right]
\label{response_matrix_eq}
\end{equation}

 \subsubsection{External current source}
 This domain condition is implemented using the last line of Eq.\,\ref{em2da_weak}. 
 
 \subsubsection{PMC (perfect magnetic conductor)}
 
 \subsubsection{PEC (perfect electric conductor/electric field BC}
  
\subsection{2D frequency domain(EM2D)}

\subsection{1D frequency domain(EM1D)}


\section{Thermal Modules}
\subsection{3D static (TH3Ds)}


\chapter{Linear system assembly}
\section{Mesh extension} 
\section{Name space}
\section{Matrix assembly}

\chapter{Solver}
\section{Setting up solver section}
This section discuss SolveStep and reduced linear system generation.

\section{Direct Solver}
\subsection{MUMPS}
MUMPS is a sparse direct solver (\url{http://mumps.enseeiht.fr}). Petra-M support MUMPS via the Petra\_MUMPS module. 
\subsection{Strumpack}
Supporting STRUMPACK is in progress.
\section{Iterative Solver}
\subsection{preconditioner}


\subsection{user defined preconditioner}
A user can also define his/her own preconditioner using python script.  This can be done in several levels, but all
uses the prc decorator to tell PetraM that a function in a script is to be used as a preconditioner object.

The first approach is to use $@$prc.block decoeator (Listing.\,\ref{prc1}). This is to define a preconditioner block element.  The defined element can be used either via a GUI discussed in the previous subsection or a user defined precontioner discussed below.  


A user can use $@$prc.blk\_diagonal and $@$prc.blk\_lowertriagular to define a diagonal and a lowertriangular precondiionter, respectively (Listing.\,\ref{prc2}).. In the decorated function, a user fill the preconditioner element using preconditioner blockblocks. Additionally, a user may create a complete precondtioner object using $@$prc.blk\_generic. In this case,SetOperator and Mult methods should be defined  by a user too.(Listing.\,\ref{prc3}). 

\begin{minipage}[c]{0.95\textwidth}
\begin{lstlisting}[caption={A user defined preconditioner block},captionpos=b, frame=single, label={prc1}]
@prc.block
def GS(**kwargs)
    prc = kwargs.pop('prc')
    blockname = kwargs.pop('blockname')
    mat = prc.get_diagoperator_by_name(blockname)
    if use_parallel:
        smoother = mfem.HypreSmoother(mat)
        ktype = getattr(mfem.HypreSmoother, "GS")
        smoother.SetType(ktype)
    else:
        smoother = mfem.GSSmoother(mat, 0)
        smoother.iterative_mode = False
    return smoother
\end{lstlisting}
\end{minipage}

\begin{minipage}[c]{0.95\textwidth}
\begin{lstlisting}[frame=single, caption={A user defined diag/lowertriangular preconditioner},captionpos=b, label={prc2}]
from petram.helper.preconditioners import prc
from petram.helper.preconditioners import GS as GSg
@prc.blk_diagonal
def D(p, g, *args, **kwargs):
    '''
    definition of block diagonal preconditioner
    p is a preconditioner object. In this case,
     mfem.BlockDiagonalPreconditioner
    g is a preconditioner generator object carrying 
    the information of matrix.
    This example set GS for unknown v.  
    '''
    GSg.set_param(g, "v")
    gs = GSg() #instatiate a block
    k = g.get_row_by_name("v")
    p.SetDiagonalBlock(k, gs) # set to block
    return p
    
@prc.blk_lowertriagular
def prc_tri(p, **kwargs):
    '''
    definition of lower trianuglar preconditioner
    '''
    p.Mult(x, y)
\end{lstlisting}
\end{minipage}

\begin{minipage}[c]{0.95\textwidth}
\begin{lstlisting}[caption={A user defined generic preconditioner},captionpos=b, frame=single, label={prc3}]
@prc.blk_generic
def G(p, **kwargs):


@G.Mult
def Mult(p, x, y):
      '''
      contents of Mult method. 
      x is input, while y is output
      '''

@G.SetOperator
def SetOperator(p, opr):
      '''
      Update preconditioner using opr.
      '''
\end{lstlisting}
 \end{minipage}


\chapter{Postprocess}
        
\section{Derived Value}     
Create a new grid function and project expression.
Extended Mesh must be generated in the solve stage.

\chapter{Visualization}
\section{Solution file structure}
\section{Plotting solution in $\pi$Scope}

\end{document}